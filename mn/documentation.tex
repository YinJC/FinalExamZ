%%%%%%%%%%%%%%%%%%%%%%%%%%%%%%%%%%%%%%%%%%%%%%%%%%%%
% 
% finalexamz - Шалгалтын материалын эх бэлтгэхэд ашиглагдах 
% finalexamz - LaTeX системд суурилсан багажийн баримтжуулалт
%
% Хувилбар 1.1.0
%
% Хөгжүүлсэн Галаа (www.galaa.mn)
% Copyright (c) 2015 Galaa
% Лиценз GPLv3 (www.gnu.org/copyleft/gpl.html)
%
%%%%%%%%%%%%%%%%%%%%%%%%%%%%%%%%%%%%%%%%%%%%%%%%%%%%


\documentclass[10pt]{article}
\usepackage[utf8]{inputenc}
\usepackage[mongolian]{babel}
\usepackage{amsthm,amsmath,amssymb,graphicx,xcolor,verbatim,hyperref,tikz}
\usepackage[a4paper,margin=3cm,left=4cm]{geometry}

\theoremstyle{definition}
\newtheorem{example}{Жишээ}

\DeclareMathOperator{\cov}{cov}

\begin{document}

\title{
  {\Huge FinalExamZ} \\[2mm] Шалгалтын материал бэлдэхэд зориулагдсан \\ \LaTeX\, загвар \& багаж \\[2mm] \large\version \\[2mm] \normalsize \svn}
\author{
  Ганболдын МАХГАЛ\\
  Монгол Улсын Их Сургууль\\
  Хэрэглээний Шинжлэх Ухаан, Инженерчлэлийн Сургууль\\
  Хэрэглээний Математикийн тэнхим\\[2mm]
  \texttt{makhgal@seas.num.edu.mn} \\[2mm]
  \www}
\newcommand{\version}{v1.1.0}
\newcommand{\www}{\url{http://www.galaa.mn/}}
\newcommand{\svn}{\url{https://github.com/galaamn/FinalExamZ}}
\date{2015 оны 4-р сарын 28}
\maketitle

\begin{abstract}
FinalExamZ бол \LaTeX\, систем дээрх шалгалтын материалын загвар ба бодлого, асуулт, сонгох болон нөхөх тест бүхий олон варианттай шалгалтын материал бэлдэх багажийн цогц нийлэмж бөгөөд энгийн текст, математикийн томъёо, схем диаграм, зураг, хүснэгт зэргийг агуулсан шалгалтын материал бэлдэх ажлыг хөнгөвчилнө.\\[2mm]
\textbf{Түлхүүр үгс:} шалгалтын материал бэлдэх, шалгалтын материалын загвар, \LaTeX
\end{abstract}

\tableofcontents

\section{Лиценз ба ашиглагдсан сангууд}\label{license}

\paragraph{Лиценз} FinalExamZ багаж ба загварыг хуулбарлах, түгээх, өөрчлөх нь LATEX Project Public License (LPPL) (\url{http://www.latex-project.org/lppl.txt}) лицензийн 1.3 ба түүнээс хойших хувилбар дахь нөхцлөөр зохицуулагдана.

\paragraph{Ашиглагдсан сангууд} FinalExamZ-ийг зохиоход amsmath, amssymb, fmtcount, graphicx, xcolor, fancyhdr, multido, ifthen, background, xstring, tikz багцууд болон ulem сангийн normalem дэд хэсэг ашиглагдсан. Мөн монгол хэлний дэмжлэгийн үүднээс inputenc ба babel сангууд дуудагдана.

\section{Бүтэц}

Энэхүү багц нь дараах нэр бүхий файлуудаас бүтнэ.
\begin{enumerate}
 \item FinalExamZ.tex
 \item FinalExamZ.pdf
 \item Variant\_1.tex
 \item Variant\_2.tex
 \item documentation.pdf
\end{enumerate}

\section{Суулгах ба ашиглах}

FinalExamZ нь загвар ба багажийн шийдлийг хамтад нь агуулсан учраас \texttt{FinalExamZ.tex} файл нь \LaTeX-ийн толгой файл болон шууд ашиглагдана. Мөн \texttt{Variant\_1.tex} файл заавал, шаардлагатай тохиолдолд \texttt{Variant\_2.tex} мэтчилэн нэмэлт файлуудыг зохих форматын дагуу үүсгэж \texttt{FinalExamZ.tex} файлын хамтаар нэг хавтсанд байрлуулсан байна.

\section{Тохиргоо}

\subsection{Ерөнхий тохиргоо}

Ерөнхий тохиргоо нь \texttt{FinalExamZ.tex} файлын эхэн хэсэгт дараах маягтайгаар байрлана.
\begin{verbatim}
\newcommand{\institution}{Сургуулийн нэр}
\newcommand{\logo}{эмблем лого}
\newcommand{\examtype}{Шалгалтын төрөл}
\newcommand{\examdate}{Шалгалтын огноо}
\newcommand{\coursetitle}{Хичээлийн нэр}
\newcommand{\coursecode}{Хичээлийн индекс}
\newcommand{\teacher}{Багшийн нэр}
\newcommand{\totaltime}{Шалгалтын хугацаа}
\newcommand{\materials}{Ашиглаж болох материалууд}
\newcommand{\caution}{Өөр бусад анхааруулга, санамж, зөвлөмж}
\newcommand{\examvariants}{вариантын тоо}
\newcommand{\printsolution}{хариуг хэвлэх эсэх}
\end{verbatim}

\subsubsection{\texttt{institution} параметр}

Уг параметр нь шалгалт авч буй сургуулийн нэрийг заахад ашиглагдана. Параметрийн утгыг жишээлбэл 
\begin{verbatim}
\newcommand{\institution}{МОНГОЛ УЛСЫН ИХ СУРГУУЛЬ}
\end{verbatim}
байдлаар зааж өгнө.

\subsubsection{\texttt{logo} параметр}

Үүнийг байгууллагын эмблем, логог шалгалтын материалын нүүрэнд  байрлуулахад ашиглана. Параметрийн утгыг 
\begin{verbatim}
\newcommand{\logo}{файлын нэр}
\end{verbatim}
байдлаар зааж өгнө. Файл нь PNG, JPG, GIF зэрэг нийтлэг форматуудын алинаар ч байж болно. Хэрвээ лого байршуулахгүй бол параметрийн утгыг 
\begin{verbatim}
\newcommand{\logo}{}
\end{verbatim}
гэж хоосноор үлдээнэ.

\subsubsection{\texttt{examtype} параметр}

Шалгалтын төрлийг заахад ашиглагдана. Жишээлбэл Улирлын шалгалт, Явцын шалгалт гэх мэт. Параметрийг тохируулах байдал 
\begin{verbatim}
\newcommand{\examtype}{Явцын шалгалт}
\end{verbatim}
маягтай байна.

\subsubsection{\texttt{examdate} параметр}

Шалгалтын огноог шалгалтын материалын нүүр хуудсанд хэвлэхэд ашиглагдана. Жишээбэл 
\begin{verbatim}
\newcommand{\examdate}{2015 оны 6 сарын 10}
\end{verbatim}
маягаар огноог дурын байдлаар өгөх боломжтой.

\subsubsection{\texttt{coursetitle} параметр}

Хичээлийн нэрийг үүгээр заана. Жишээбэл 
\begin{verbatim}
\newcommand{\coursetitle}{Магадлалын онол}
\end{verbatim}

\subsubsection{\texttt{coursecode} параметр}

Энэ параметрээр тухайн хичээлийн кодыг заана. Жишээбэл
\begin{verbatim}
\newcommand{\coursecode}{PROS211}
\end{verbatim}

\subsubsection{\texttt{teacher} параметр}

Хичээл заасан багшийн нэр, шалгалт авч буй хүний нэрийг энд өгнө. Жишээбэл
\begin{verbatim}
\verb|\newcommand{\teacher}{Г.Махгал, Ш.Мөнгөнсүх}|
\end{verbatim}

\subsubsection{\texttt{totaltime} параметр}

Шалгалтын хугацааг энд өгнө. Хугацааг дурын байдлаар заах боломжтой. Жишээбэл
\begin{verbatim}
\newcommand{\totaltime}{2 цаг}
\end{verbatim}
эсвэл
\begin{verbatim}
\newcommand{\totaltime}{120 минут}
\end{verbatim}
гэх мэт.

\subsubsection{\texttt{materials} параметр}

Шалгалтын үеэр ашиглаж болох материалуудыг заахад ашиглагдана. Жишээбэл \begin{verbatim}
\newcommand{\materials}{Тооны машин, тархалтын хүснэгтүүд, 
томъёоны хураамж}
\end{verbatim}
эсвэл
\begin{verbatim}
\newcommand{\materials}{байхгүй}
\end{verbatim}
гэх мэт.

\subsubsection{\texttt{caution} параметр}

Анхааруулга, санамж зэргийг энэ параметрээр дамжуулан оруулна. Жишээбэл 
\begin{verbatim}
\newcommand{\caution}{Гар утас, талблет, компьютер ашиглах, хуулах, 
бусдад саад учруулах аливаа үйлдлийг хориглоно.}
\end{verbatim}
Хэрвээ шаардлагагүй бол хоосон үлдээж болно. Ийм тохиолдолд автоматаар уг талбарыг алгасах болно.

\subsubsection{\texttt{examvariants} параметр}

Шалгалтын вариантын тоог энд зааж өгнө. Жишээбэл 
\begin{verbatim}
\newcommand{\examvariants}{4}
\end{verbatim}
Шалгалтын вариантууд нь \texttt{Variant\_1.tex},\ldots,\texttt{Variant\_4.tex} гэх мэтээр нэрлэгдсэн файлуудад байршихыг анхаарна уу. Өөрөөр хэлбэл файлын тоо ба параметрийн утга зөрж болохгүй гэсэн үг юм. Мөн FinalExamZ нь эдгээр файлуудыг автоматаар дуудаж уншихаар програмчлагдсан.

\subsubsection{\texttt{printsolution} параметр}

Бодолт, хариу, түлхүүрийг материал дээр хэвлэх эсэхийг заана. Уг параметр нь \texttt{yes} ба \texttt{no} утгуудын аль нэгийг авна. Жишээбэл 
\begin{verbatim}
\newcommand{\printsolution}{yes}
\end{verbatim}
Бодолт ба хариуг хэрхэн зааж өгөхийг \ref{tools} хэсгээс үзнэ үү.

\subsection{Бусад тохиргоо}

Энд бодолт ба хариултын хэсгийн хэмжээг заах зэрэг тусгай параметрүүд орно. Үүнийг \ref{tools} хэсгээс үзнэ үү.

\section{Загвар}

FinalExamZ нь үндсэн нэг загвартай бөгөөд загвар дизайныг өөрчлөх, хуулбарлах нөхцлийн талаар \ref{license} хэсэг бичсэн болно. Загварыг зохиохдоо гадаадын зарим их сургуулийн шалгалтын материалын загварыг харгалзан үзсэн.

\par Цаасны хэмжээ A4, маржин (дээрээс, баруунаас, дороос, зүүнээс) 2.5 см, 2 см, 2.5 см, 3 см, үсгийн өндөр 10pt нэгжээр авагдсан. Тохиргоонд ашиглагдах уртын хэмжээсийн нэгжийг сантиметрээр авсан.

\par Шалгалтын материалыг бүх вариантаар нэг бүхэл файл болгон гаргах бөгөөд ингэхдээ хуудасны дугаарыг нийт хуудасны (тухайн вариантын хувьд) хамтаар тооцон хуудас бүрийн хөл хэсэгт хэвлэхээр програмчилж өгсөн.

\section{Багаж}\label{tools}

FinalExamZ нь
\begin{itemize}
 \item бодлого
 \item асуулт
 \item сонгох тест
 \item нөхөх тест
\end{itemize}
оруулахад ашиглагдах дөрвөн үндсэн багаж буюу функцтэй. Үүнд:
\begin{itemize}
 \item \verb|\problem[бодолтын хэсгийн өндрийн хэмжээ]{өгүүлбэр}{оноо}{бодолт}|
 \item \verb|\question[хариултын хэсгийн өндрийн хэмжээ]{асуулт}{оноо}{хариу}|
 \item \verb|\stest{асуулт}{оноо}{сонголт1//[+]зөвсонголт//сонголт3}|
 \item \verb|\ptest{асуулт \ehide{томъёо} асуулт \thide{текст} асуулт}{оноо}|
\end{itemize}
зэрэг болно. 

\paragraph{Үндсэн өгөгдөхүүнүүд} Дээрх функцүүдийн \texttt{\{\}} хаалтанд харгалзах аргументүүдийг заавал өгөх ёстойг анхаарна уу. Жишээлбэл оноог заавал заасан байна. 

\paragraph{Оноо} Оноог заавал өгөхөөс гадна зөвхөн тоогоор оруулна. Энэхүү оноо нь бодлого, асуулт, тест нэг бүрчлэн хэвлэгдэх бөгөөд улмаар FinalExamZ нь шалгалтын материалын төгсгөлд нийт оноог вариант тус бүрээр автоматаар тооцож гаргахаар програмчлагдсан.

\paragraph{Бодолт ба хариултын хэсгийн өндрийн хэмжээ} Бодолт ба хариултын хэсгийн өндрийн хэмжээ буюу \texttt{[]} хаалтан дахь аргументын утгыг шаардлагатай үедээ зөвхөн тоогоор өгнө. Өөрөөр хэлбэл заавал зааж өгөх албагүй бас тоон бус утга оруулж болохгүй юм. Хэрвээ уг аргументийн утгыг өгөхгүй өөрөөр хэлбэл дөрвөлжин хаалтыг хоосон үлдээхээр бол тухайн (хос) дөрвөлжин хаалтыг заавал устгана. Уг хэмжээ нь \LaTeX болон FinalExamZ-ийн програмчлал зэрэг шалтгаануудын улмаас эцсийн үр дүн (pdf файл) дээрээ заагдсан хэмжээтэйгээ яг адилхан байх албагүйг анхаарна уу.

\par Эдгээрээс гадна дараах хэсгүүдэд заагдсан дүрмүүдийг баримтлах шаардлагатай.

\subsection{Бодлого}

Бодлого оруулах багаж буюу функц дараах маягтай.
\begin{verbatim}
\problem[бодолтын хэсгийн өндрийн хэмжээ]{өгүүлбэр}{оноо}{бодолт}
\end{verbatim}

\begin{example}[Бодлого оруулах байдал]
5 оноотой магадлалын онолын нэгэн бодлогыг бодолтын хамтаар оруулахаар бичсэн кодыг дор харууллаа. Мөн бодолтын хэсгийн өндрийг 2.5 см гэж заалаа.
\begin{verbatim}
\problem[2.5]{
$(X,Y)$ вектор санамсаргүй хувьсагчийн хамтын тархалтын хууль 
дараах хүснэгтээр өгөгдөв.
\begin{center}
 \begin{tabular}{r|rrr}
  & \multicolumn{3}{c}{$Y$} \\
  $X$  & $-1$ & 0 & 1 \\
  \hline
  $-1$ & 0.2 & 0.2 & 0.1 \\
  0 & 0 & 0.1 & 0.2 \\
  1 & 0 & 0 & 0.2 \\
 \end{tabular}
\end{center}
$X$ ба $Y$ санамсаргүй хувьсагчид хамааралтай эсэхийг тогтоо.
}{
5
}{
$$P(X=-1,Y=-1)=0.2\neq 0.5\cdot 0.2=P(X=-1)\cdot P(Y=-1)$$
учраас $X$ ба $Y$ хамааралтай.
}
\end{verbatim}
\end{example}

\subsection{Асуулт}

Асуулт оруулах багаж буюу функц дараах маягтай.
\begin{verbatim}
\question[хариултын хэсгийн өндрийн хэмжээ]{асуулт}{оноо}{хариу}
\end{verbatim}

\begin{example}[Асуулт оруулах байдал]
TikZ сан ашиглан зурсан функцийн график бүхий 3 оноотой асуултыг хэрхэн оруулсныг харна уу.
\begin{verbatim}
\question{
Гурван өөр тархалтын нягтын функцийн график өгөгдөв.
\begin{center}
 \begin{tikzpicture}[xscale=1,yscale=5]
  \draw[->] (-2.2,0) -- (2.2,0);
  \draw[->] (-1.8,-0.1) -- (-1.8,0.5);
  \draw[] plot[domain=-2:2] (\x,{1/sqrt(2*pi)*exp(-1*\x*\x/2)});
  \draw[dashed] plot[domain=-2:2] (\x,{1/(pi*(1+\x*\x))});
  \draw[dotted] plot[domain=-2:2] (\x,{exp(-sqrt(\x*\x))/2});
 \end{tikzpicture}
\end{center}
Тэдгээр тархалтуудын аль илүү их эксцесстэй вэ?
}{
3
}{
Цэгээр зурагдсан графикт харгалзах тархалтын эксцесс 
бусдаасаа их байна. Учир нь орой нь илүү шовх бас 
тархалтын орой шовх болох тусам эксцесс өсдөг чанартай.
}
\end{verbatim}
\end{example}

\subsection{Сонгох тест}

Сонгох тестийг дараах форматаар оруулна.
\begin{verbatim}
\stest{асуулт}{оноо}{сонголт1//[+]зөвсонголт//сонголт3}
\end{verbatim}
Тестийн хариултуудыг \texttt{//} тэмдгээр тусгаарлах бол харин зөв хариултыг дээрхийн адилаар \texttt{[+]} тэмдгээр тодотгож өгнө. Хэрвээ тест олон сонголттой өөрөөр хэлбэл олон зөв хариулттай бол сонголт бүрт \texttt{[+]} тэмдгийг оруулна.
\begin{verbatim}
\stest{асуулт}{оноо}{[+]зөвсонголт1//[+]зөвсонголт2//сонголт3}
\end{verbatim}

\begin{example}[Сонгох тест оруулах байдал]
Нэг нь зөв, 4 хувилбар бүхий сонголттой тестийг оруулах кодын бичлэгийг сонирхоно уу.
\begin{verbatim}
\stest{
Моод ямар төрлийн тоон үзүүлэлт вэ?
}{
2
}{
[+]Төвийн//Хазайлтын//Тархалтын хэлбэрийн//Хамаарлын
}
\end{verbatim}
\end{example}

\subsection{Нөхөх тест}

Нөхөх тестийг дараах форматаар оруулна.
\begin{verbatim}
\ptest{асуулт \ehide{томъёо} асуулт \thide{текст} асуулт}{оноо}
\end{verbatim}
Нөхөх тестийг шивж оруулахдаа тестийн агуулгаа бүрэн бичих бөгөөд нөхүүлж бичүүлэх хэсгээ нуухдаа FinalExamZ дээр тодорхойлогдсон тусгай функцүүд болох \verb|\ehide{}| ба \verb|\thide{}| функцүүдийг ашиглана.

\begin{example}[Нөхөх тест оруулах байдал]
Томъёо нөхөж бичих тест дээр \verb|\ehide{}| функцийг ашигласныг харна уу.
\begin{verbatim}
\ptest{
$EX=2$ ба $EY=1$ байв. Тэгвэл $E(2X+Y)=\ehide{2EX}+EY$.
}{
3
}
\end{verbatim}
Харин текст нөхөж бичих тестийн хувьд \verb|\thide{}| функцийг ашиглалаа.
\begin{verbatim}
\ptest{
Характеристик функцийн ашиглан тархалтын хуулийг олохдоо 
характеристик функцийн \thide{урвалтын томъёо}г хэрэглэдэг.
}{
3
}
\end{verbatim}
\end{example}

Эцэст нь \verb|\thide{}| функц олон мөр дамнасан текстийг нууж чаддаг болохыг тэмдэглэе.

\begin{center}
 $\ast\ast\ast$ Төгсгөл $\ast\ast\ast$
\end{center}

\end{document}
