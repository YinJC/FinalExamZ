%%%%%%%%%%%%%%%%%%%%%%%%%%%%%%%%%%%%%%%%%%%%%%%%%%%%
% 
% finalexamz - Шалгалтын материалын эх бэлтгэхэд ашиглагдах 
% finalexamz - LaTeX системд суурилсан багажийн баримтжуулалт
%
% Хөгжүүлсэн Галаа (www.galaa.mn)
% Copyright (c) 2015-2016 Galaa
% Лиценз GPLv3 (www.gnu.org/copyleft/gpl.html)
%
%%%%%%%%%%%%%%%%%%%%%%%%%%%%%%%%%%%%%%%%%%%%%%%%%%%%


\documentclass[10pt]{article}
\usepackage[utf8]{inputenc}
\usepackage[mongolian]{babel}
\usepackage{amsthm,amsmath,amssymb,graphicx,xcolor,verbatim,hyperref,tikz}
\usepackage[a4paper,margin=3cm,left=4cm]{geometry}

\hypersetup{
  unicode=true
}

\theoremstyle{definition}
\newtheorem{example}{Жишээ}

\DeclareMathOperator{\cov}{cov}

\begin{document}

\title{
  {\Huge FinalExamZ} \\[2mm] Шалгалтын материал бэлдэхэд зориулагдсан \\ \LaTeX\, загвар \& багаж \\[2mm] \large\version \\[2mm] \normalsize \svn}
\author{
  Ганболдын МАХГАЛ\\
  Монгол Улсын Их Сургууль\\
  Хэрэглээний Шинжлэх Ухаан, Инженерчлэлийн Сургууль\\
  Хэрэглээний Математикийн тэнхим\\[2mm]
  \texttt{makhgal@seas.num.edu.mn} \\[2mm]
  \www}
\newcommand{\version}{v1.4.0 -- 2016/12/24}
\newcommand{\www}{\url{http://www.galaa.mn/}}
\newcommand{\svn}{\url{https://github.com/galaamn/FinalExamZ}}
\date{2015 оны 4-р сарын 28} % класс болон баримтжуулалтыг олон нийтэд анхлан түгээсэн огноо
\maketitle

\begin{abstract}
FinalExamZ бол шалгалтын материалын загвар ба бодлого, асуулт, сонгох болон нөхөх тест бүхий олон варианттай шалгалтын материал бэлдэх багажийн цогц нийлэмж -- \LaTeX\, класс бөгөөд энгийн текст, математикийн томъёо, схем диаграм, зураг, хүснэгт ба програмын эх код зэргийг агуулсан шалгалтын материал бэлдэхэд ашиглагдана.\\[2mm]
\textbf{Түлхүүр үгс:} шалгалтын материал бэлдэх, шалгалтын материалын загвар, \LaTeX
\end{abstract}

\tableofcontents

\section{Лиценз}\label{license}

FinalExamZ багаж ба загварыг хуулбарлах, түгээх, өөрчлөх нь LATEX Project Public License (LPPL) (\url{http://www.latex-project.org/lppl.txt}) лицензийн 1.3 ба түүнээс хойших хувилбар дахь нөхцлөөр зохицуулагдана.

\section{Ашиглагдсан сангууд}

FinalExamZ нь \LaTeX-ийн стандарт класс болох \texttt{article} класст тулгуурласан.

\par Уг классыг зохиоход amsmath, amssymb, textcomp, fmtcount, graphicx, xcolor, fancyhdr, multido, ifthen, background, xstring, tikz, listings багцууд болон ulem сангийн normalem дэд хэсэг ашиглагдсан. Мөн монгол хэлний дэмжлэгийн үүднээс inputenc ба babel сангууд дуудагдана.

\section{Бүтэц}

Энэхүү багц нь дараах нэр бүхий файлуудаас бүтнэ.
\begin{enumerate}
 \item Exam.pdf (үр дүн)
 \item Exam.tex (толгой файл)
 \item finalexamz.cls (\LaTeX\ class)
 \item Variant\_1.tex (шалгалтын бодлогууд)
 \item Variant\_2.tex (шалгалтын бодлогууд)
 \item documentation.pdf (баримтжуулалт)
\end{enumerate}

\section{Хувилбарууд дахь өөрчлөлтийн бүртгэл}

\subsection{2015 оны 4 сарын 28, хувилбар 1.0.0}

\begin{enumerate}
 \item Анхны хувилбар
\end{enumerate}

\subsection{2015 оны 4 сарын 30, хувилбар 1.1.0}

\begin{enumerate}
 \item "Хариуг хэвлэх горим"{}-д "Анхааруулга"\ хэсгийг алгасдаг болсон.
 \item Багцын нэрийг хажуугийн багананд нэмсэн.
 \item Хажуугийн багана дахь үг хоорондын зай өөрчлөгдсөн.
\end{enumerate}

\subsection{2016 оны 3 сарын 13, хувилбар 1.2.0}

\begin{enumerate}
 \item Нөхөх тестийн текстэн хариуг налуугаар хэвлэх болсон.
 \item Нөхөх тестийн томъёон хариуг доогуур зураастайгаар хэвлэх болсон.
 \item Бодолт ба хариултын хэсгийн өндрийг заах явдал тогворжсон.
 \item Оюутны нэр ба ID зэргийг бичих тусгай талбар хуудасны толгой хэсэгт нэмэгдсэн.
 \item Програмын эх кодыг оруулах явдал хялбарчлагдсан.
\end{enumerate}

\subsection{2016 оны 3 сарын 24, хувилбар 1.3.0}

\begin{enumerate}
 \item Класс хэлбэрт шилжив.
 \item Зарим чухал бус талбарын утгыг хоосноор зааж өгөх боломжийг бүрдүүлэв.
 \item Бодолт ба хариултын хэсгийн өндрийг заах явдал дахь алдааг засав.
 \item Вариантуудыг оруулах байдлыг өөрчилж environment уруу шилжүүлэв.
 \item Нүүр хуудас ба анхааруулгыг хэвлэх эсэхийг тусгай командаар удирдах боломжийг оруулав.
\end{enumerate}

\subsection{2016 оны 12 сарын 24, хувилбар 1.4.0}

\begin{enumerate}
 \item Сонгох тестүүд дэх сонголтын дугаар ба текст хооронд шинэ мөр эхлэх явдлыг хориглов.
\end{enumerate}

\section{Суулгах}

FinalExamZ классыг агуулж буй \texttt{finalexamz.cls} файлыг баримтын толгой файл байрлаж буй хавтсанд хуулах бөгөөд улмаар \verb|\documentclass{finalexamz}| маягаар баримтын классыг зааж өгнө.

\section{Тохиргоо}

\subsection{Монгол хэлний дэмжлэг}

Классыг \texttt{mongolian} параметртэйгээр дуудна. Тодруулбал,
\begin{verbatim}
\documentclass[mongolian]{finalexamz}
\end{verbatim}
Үүний дүнд монгол хэлний \texttt{babel} багцыг дуудахаас гадна загвар дахь хэлний хувьсагчдын утгыг монголоор оноох болно.

\subsection{Классын тохиргоо}

Уг класс нь
\begin{enumerate}
 \item хариуг хэвлэх
 \item үсгийн хэмжээг өөрчлөх
 \item ноорог байдлаар боловсруулах
\end{enumerate}
зэргийг заах тусгай параметрүүдтэй. Олон параметрийг зэрэг өгөх тохиолдолд параметрүүдийг таслалаар тусгаарлана.

\subsection{Хариуг хэвлэх}

Классыг \texttt{solution} параметртэйгээр дуудна. \\ Тодруулбал, \verb|\documentclass[solution]{finalexamz}|

\subsection{Үсгийн хэмжээг өөрчлөх}

Үсгийн хэмжээ pt хэмжээгээр 10, 11 ба 12 гэх гурван сонголттой. Жишээлбэл үсгийн өндрийг 11pt хэмжээтэйгээр заахдаа классыг \verb|\documentclass[11pt]{finalexamz}| гэж дуудна.

\subsection{Нооргийн горимд шилжүүлэх}

Баримтыг ноорог буюу draft хэлбэрээр боловсруулахын тулд \texttt{draft} параметрийг нэмж өгнө. Тодруулбал, \verb|\texttt{\documentclass[draft]{finalexamz}|

\subsection{Ерөнхий мэдээлэл}

Ерөнхий тохиргоо нь баримтын толгой файлын эхэн хэсэгт дараах маягтайгаар байрлана.
\begin{verbatim}
\def\institution{Сургуулийн нэр}
\def\logo{эмблем лого}
\def\examtype{Шалгалтын төрөл}
\def\examdate{Шалгалтын огноо}
\def\coursetitle{Хичээлийн нэр}
\def\coursecode{Хичээлийн индекс}
\def\teacher{Багшийн нэр}
\def\totaltime{Шалгалтын хугацаа}
\def\materials{Ашиглаж болох материалууд}
\def\caution{Нэмэлт анхааруулга, санамж, зөвлөмж}
\end{verbatim}

\subsubsection{\texttt{institution} параметр}

Уг параметр нь шалгалт авч буй сургуулийн нэрийг заахад ашиглагдана. Параметрийн утгыг жишээлбэл 
\begin{verbatim}
\def\institution{МОНГОЛ УЛСЫН ИХ СУРГУУЛЬ}
\end{verbatim}
байдлаар зааж өгнө.

\subsubsection{\texttt{logo} параметр}

Үүнийг байгууллагын эмблем, логог шалгалтын материалын нүүрэнд  байрлуулахад ашиглана. Параметрийн утгыг 
\begin{verbatim}
\def\logo{файлын нэр}
\end{verbatim}
байдлаар зааж өгнө. Файл нь PNG, JPG, GIF зэрэг нийтлэг форматуудын алинаар ч байж болно. Хэрвээ лого байршуулахгүй бол параметрийн утгыг 
\begin{verbatim}
\def\logo{}
\end{verbatim}
гэж хоосноор үлдээнэ.

\subsubsection{\texttt{examtype} параметр}

Шалгалтын төрлийг заахад ашиглагдана. Жишээлбэл Улирлын шалгалт, Явцын шалгалт гэх мэт. Параметрийг тохируулах байдал 
\begin{verbatim}
\def\examtype{Явцын шалгалт}
\end{verbatim}
маягтай байна.

\subsubsection{\texttt{examdate} параметр}

Шалгалтын огноог шалгалтын материалын нүүр хуудсанд хэвлэхэд ашиглагдана. Жишээбэл 
\begin{verbatim}
\def\examdate{2015 оны 6 сарын 10}
\end{verbatim}
маягаар огноог дурын байдлаар өгөх боломжтой.

\subsubsection{\texttt{coursetitle} параметр}

Хичээлийн нэрийг үүгээр заана. Жишээбэл 
\begin{verbatim}
\def\coursetitle{Магадлалын онол}
\end{verbatim}

\subsubsection{\texttt{coursecode} параметр}

Энэ параметрээр тухайн хичээлийн кодыг заана. Жишээбэл
\begin{verbatim}
\def\coursecode{PROS211}
\end{verbatim}
Мөн уг параметрийг хоосноор заах боломжтой. Энэ тохиолдолд хичээлийн кодтой холбогдох зүйлсийг хэвлэх явдлыг алгасах болно.

\subsubsection{\texttt{teacher} параметр}

Хичээл заасан багшийн нэр, шалгалт авч буй хүний нэрийг энд өгнө. Жишээбэл
\begin{verbatim}
\def\teacher{Г.Махгал, Ш.Мөнгөнсүх}
\end{verbatim}

\subsubsection{\texttt{totaltime} параметр}

Шалгалтын хугацааг энд өгнө. Хугацааг дурын байдлаар заах боломжтой. Жишээбэл
\begin{verbatim}
\def\totaltime{2 цаг}
\end{verbatim}
эсвэл
\begin{verbatim}
\def\totaltime{120 минут}
\end{verbatim}
гэх мэт.

\subsubsection{\texttt{materials} параметр}

Шалгалтын үеэр ашиглаж болох материалуудыг заахад ашиглагдана. Жишээбэл
\begin{verbatim}
\def\materials{Тооны машин, тархалтын хүснэгтүүд, томъёоны хураамж}
\end{verbatim}
эсвэл
\begin{verbatim}
\def\materials{}
\end{verbatim}
гэх мэт.

\subsubsection{\texttt{caution} параметр}

Анхааруулга, санамж зэргийг энэ параметрээр дамжуулан оруулна. Жишээбэл 
\begin{verbatim}
\def\caution{Гар утас, талблет, компьютер ашиглах, хуулах, 
бусдад саад учруулах аливаа үйлдлийг хориглоно.}
\end{verbatim}
Хэрвээ шаардлагагүй бол хоосон үлдээж болно. Ийм тохиолдолд автоматаар уг талбарыг алгасах болно.

\subsection{Бусад тохиргоо}

Энд бодолт ба хариултын хэсгийн хэмжээг заах зэрэг тусгай параметрүүд орно. Үүнийг \ref{tools} хэсгээс үзнэ үү.

\section{Загвар}

FinalExamZ нь үндсэн нэг загвартай бөгөөд загвар дизайныг өөрчлөх, хуулбарлах нөхцлийн талаар \ref{license} хэсэг бичсэн болно. Загварыг зохиохдоо гадаадын зарим их сургуулийн шалгалтын материалын загварыг харгалзан үзсэн.

\par Цаасны хэмжээ A4, маржин (дээрээс, баруунаас, дороос, зүүнээс) 2.5 см, 2 см, 2.5 см, 3 см, үсгийн өндөр 10pt, 11pt ба 12pt нэгжээр авагдсан. Үсгийн өндрийг классын параметрээр дамжуулан удирдах боломжтой. Тохиргоонд ашиглагдах уртын хэмжээсийн нэгжийг сантиметрээр авсан.

\par Шалгалтын материалыг бүх вариантаар нэг бүхэл файл болгон гаргах бөгөөд ингэхдээ хуудасны дугаарыг нийт хуудасны (тухайн вариантын хувьд) хамтаар тооцон хуудас бүрийн хөл хэсэгт хэвлэхээр програмчилж өгсөн.

\subsection{Нүүр хуудас}

Нүүр хуудсыг баримтандаа оруулахын тулд \verb|\coverpage| командыг өгнө.

\subsection{Анхааруулга хэсэг}

Вариантын эхэнд байх анхааруулгыг \verb|\attention| командаар дуудна.

\section{Багаж}\label{tools}

\subsection{Ерөнхий зүйл}

FinalExamZ нь
\begin{itemize}
 \item бодлого
 \item асуулт
 \item сонгох тест
 \item нөхөх тест
\end{itemize}
оруулахад ашиглагдах дөрвөн үндсэн багаж буюу функцтэй. Үүнд:
\begin{itemize}
 \item \verb|\problem[бодолтын хэсгийн өндрийн хэмжээ]{өгүүлбэр}{оноо}{бодолт}|
 \item \verb|\question[хариултын хэсгийн өндрийн хэмжээ]{асуулт}{оноо}{хариу}|
 \item \verb|\stest{асуулт}{оноо}{сонголт1//[+]зөвсонголт//сонголт3}|
 \item \verb|\ptest{асуулт \ehide{томъёо} асуулт \thide{текст} асуулт}{оноо}|
\end{itemize}
зэрэг болно. 

\subsubsection{Үндсэн өгөгдөхүүнүүд} Дээрх функцүүдийн \texttt{\{\}} хаалтанд харгалзах аргументүүдийг заавал өгөх ёстойг анхаарна уу. Жишээлбэл оноог заавал заасан байна. 

\subsubsection{Оноо} Оноог заавал өгөхөөс гадна зөвхөн тоогоор оруулна. Энэхүү оноо нь бодлого, асуулт, тест нэг бүрчлэн хэвлэгдэх бөгөөд улмаар FinalExamZ нь шалгалтын материалын төгсгөлд нийт оноог вариант тус бүрээр автоматаар тооцож гаргахаар програмчлагдсан.

\subsubsection{Бодолт ба хариултын хэсгийн өндрийн хэмжээ} Бодолт ба хариултын хэсгийн өндрийн хэмжээ буюу \texttt{[]} хаалтан дахь аргументын утгыг шаардлагатай үедээ зөвхөн тоогоор өгнө. Өөрөөр хэлбэл заавал зааж өгөх албагүй бас тоон бус утга оруулж болохгүй юм. Хэрвээ уг аргументийн утгыг өгөхгүй өөрөөр хэлбэл дөрвөлжин хаалтыг хоосон үлдээхээр бол тухайн (хос) дөрвөлжин хаалтыг заавал устгана. Уг хэмжээ нь \LaTeX\ болон FinalExamZ-ийн програмчлал зэрэг шалтгаануудын улмаас эцсийн үр дүн (pdf файл) дээрээ заагдсан хэмжээтэйгээ яг адилхан байх албагүйг анхаарна уу.

\subsubsection{Програмын эх код оруулах} Програмын кодыг оруулахтай холбоотойгоор listings багцыг санал болгож байна. Кодыг оруулах явдал нь тухайн кодын текст дунд байрлах (inline) эсвэл дангаараа мөр эзэлж байрлахаас ихээхэн хамааралтай.

\paragraph{Текст дунд байрлах кодыг оруулах} Үүний тулд
\begin{verbatim}
\lstinline[language=R,keywords={mean}]|mean(1:100)|
\end{verbatim}
маягаар кодоо бичиж өгнө.

\paragraph{Дангаараа мөр эзлэн байрлах кодыг оруулах} Энд үндсэн хоёр аргыг авч үзье.
\begin{enumerate}
\item Эхний арга бол кодыг тусгай командын тусламжтайгаар дараах маягаар урьдчилж зааж өгөх явдал юм.
\begin{verbatim}
\begin{lrbox}{\lstListing}
\begin{lstlisting}[language=python]
import random, math

c = 2.2039

while True :
  u = random.random()
  y = -1.0 * math.log(random.random())
  if c * u < y * (math.exp(-1.0 * y ** 2 / 2) + y) :
    print y
    break
\end{lstlisting}
\end{lrbox}
\end{verbatim}
Харин дараа нь бодлого, асуулт, тестийнхээ агуулга дотор эх кодоо байрлуулах хэсэгтээ \verb|\printlisting| командыг бичнэ.
\item Удаах арга бол кодынхоо мөр нэг бүрт \verb|^^J| тэмдэгтүүдийг нэмж өгөх явдал юм.
Жишээлбэл
\begin{verbatim}
\begin{lstlisting}[language=R]
^^JX = 1:100
^^Jmean(X)
\end{lstlisting}
\end{verbatim}
Энэ тохиолдолд \verb|{| ба \verb|}| бас \verb|%| зэрэг тэмдэгтүүдийг оруулахдаа тухайн тэмдэгтийн өмнө \verb|\| тэмдэгтийг залгаж бичнэ. Тодруулбал
\begin{verbatim}
\begin{lstlisting}[language=R]
^^Jfor (i in 1:100) \{
^^J  if (i \%\% 2 == 0) print(i)
^^J\}
\end{lstlisting}
\end{verbatim}
\end{enumerate}

\paragraph{Код оруулахтай холбогдох нийтлэг зүйлс} Ашигласан програмчлалын хэлнийхээ нэрийг \texttt{language} параметрийн ард заана. Дэмжигдсэн хэлнүүдийн жагсаалтыг \url{https://en.wikibooks.org/wiki/LaTeX/Source_Code_Listings#Supported_languages} веб хуудаснаас үзэх боломжтой. Үүнээс гадна эх код дотор латинаас өөр үсэг тодруулбал кирилл үсэг орох боломжгүйг анхаарна уу. Харин түлхүүр үгс ба функцийн нэрсийг listings багцын keywords, morekeywords, otherkeywords, deletekeywords параметрүүдийн тусламжтайгаар зааж удирдах боломжтой.

\vspace{\baselineskip}

\par\noindent Эдгээрээс гадна дараах хэсгүүдэд заагдсан дүрмүүдийг баримтлах шаардлагатай.

\subsection{Бодлого}

Бодлого оруулах багаж буюу функц дараах маягтай.
\begin{verbatim}
\problem[бодолтын хэсгийн өндрийн хэмжээ]{өгүүлбэр}{оноо}{бодолт}
\end{verbatim}

\begin{example}[Бодлого оруулах байдал]
5 оноотой магадлалын онолын нэгэн бодлогыг бодолтын хамтаар оруулахаар бичсэн кодыг дор харууллаа. Мөн бодолтын хэсгийн өндрийг 2.5 см гэж заалаа.
\begin{verbatim}
\problem[2.5]{
$(X,Y)$ вектор санамсаргүй хувьсагчийн хамтын тархалтын хууль 
дараах хүснэгтээр өгөгдөв.
\begin{center}
 \begin{tabular}{r|rrr}
  & \multicolumn{3}{c}{$Y$} \\
  $X$  & $-1$ & 0 & 1 \\
  \hline
  $-1$ & 0.2 & 0.2 & 0.1 \\
  0 & 0 & 0.1 & 0.2 \\
  1 & 0 & 0 & 0.2 \\
 \end{tabular}
\end{center}
$X$ ба $Y$ санамсаргүй хувьсагчид хамааралтай эсэхийг тогтоо.
}{
5
}{
$$P(X=-1,Y=-1)=0.2\neq 0.5\cdot 0.2=P(X=-1)\cdot P(Y=-1)$$
учраас $X$ ба $Y$ хамааралтай.
}
\end{verbatim}
\end{example}

\subsection{Асуулт}

Асуулт оруулах багаж буюу функц дараах маягтай.
\begin{verbatim}
\question[хариултын хэсгийн өндрийн хэмжээ]{асуулт}{оноо}{хариу}
\end{verbatim}

\begin{example}[Асуулт оруулах байдал]
TikZ сан ашиглан зурсан функцийн график бүхий 3 оноотой асуултыг хэрхэн оруулсныг харна уу.
\begin{verbatim}
\question{
Гурван өөр тархалтын нягтын функцийн график өгөгдөв.
\begin{center}
 \begin{tikzpicture}[xscale=1,yscale=5]
  \draw[->] (-2.2,0) -- (2.2,0);
  \draw[->] (-1.8,-0.1) -- (-1.8,0.5);
  \draw[] plot[domain=-2:2] (\x,{1/sqrt(2*pi)*exp(-1*\x*\x/2)});
  \draw[dashed] plot[domain=-2:2] (\x,{1/(pi*(1+\x*\x))});
  \draw[dotted] plot[domain=-2:2] (\x,{exp(-sqrt(\x*\x))/2});
 \end{tikzpicture}
\end{center}
Тэдгээр тархалтуудын аль илүү их эксцесстэй вэ?
}{
3
}{
Цэгээр зурагдсан графикт харгалзах тархалтын эксцесс 
бусдаасаа их байна. Учир нь орой нь илүү шовх бас 
тархалтын орой шовх болох тусам эксцесс өсдөг чанартай.
}
\end{verbatim}
\end{example}

\subsection{Сонгох тест}

Сонгох тестийг дараах форматаар оруулна.
\begin{verbatim}
\stest{асуулт}{оноо}{сонголт1//[+]зөвсонголт//сонголт3}
\end{verbatim}
Тестийн хариултуудыг \texttt{//} тэмдгээр тусгаарлах бол харин зөв хариултыг дээрхийн адилаар \texttt{[+]} тэмдгээр тодотгож өгнө. Хэрвээ тест олон сонголттой өөрөөр хэлбэл олон зөв хариулттай бол сонголт бүрт \texttt{[+]} тэмдгийг оруулна.
\begin{verbatim}
\stest{асуулт}{оноо}{[+]зөвсонголт1//[+]зөвсонголт2//сонголт3}
\end{verbatim}

\begin{example}[Сонгох тест оруулах байдал]
Нэг нь зөв, 4 хувилбар бүхий сонголттой тестийг оруулах кодын бичлэгийг сонирхоно уу.
\begin{verbatim}
\stest{
Моод ямар төрлийн тоон үзүүлэлт вэ?
}{
2
}{
[+]Төвийн//Хазайлтын//Тархалтын хэлбэрийн//Хамаарлын
}
\end{verbatim}
\end{example}

\subsection{Нөхөх тест}

Нөхөх тестийг дараах форматаар оруулна.
\begin{verbatim}
\ptest{асуулт \ehide{томъёо} асуулт \thide{текст} асуулт}{оноо}
\end{verbatim}
Нөхөх тестийг шивж оруулахдаа тестийн агуулгаа бүрэн бичих бөгөөд нөхүүлж бичүүлэх хэсгээ нуухдаа FinalExamZ дээр тодорхойлогдсон тусгай функцүүд болох \verb|\ehide{}| ба \verb|\thide{}| функцүүдийг ашиглана.

\begin{example}[Нөхөх тест оруулах байдал]
Томъёо нөхөж бичих тест дээр \verb|\ehide{}| функцийг ашигласныг харна уу.
\begin{verbatim}
\ptest{
$EX=2$ ба $EY=1$ байв. Тэгвэл $E(2X+Y)=\ehide{2EX}+EY$.
}{
3
}
\end{verbatim}
Харин текст нөхөж бичих тестийн хувьд \verb|\thide{}| функцийг ашиглалаа.
\begin{verbatim}
\ptest{
Характеристик функцийн ашиглан тархалтын хуулийг олохдоо 
характеристик функцийн \thide{урвалтын томъёо}г хэрэглэдэг.
}{
3
}
\end{verbatim}
\end{example}

Эцэст нь \verb|\thide{}| функц олон мөр дамнасан текстийг нууж чаддаг болохыг тэмдэглэе.

\begin{center}
 $\ast\ast\ast$ Төгсгөл -- \today\ $\ast\ast\ast$
\end{center}

\end{document}
